%\wikiskriptum{01DIFRnew}
\documentclass[oneside,intlimits,reqno]{scrbook}
 
%\usepackage{eucal}
\usepackage{array}
\usepackage{enumerate}
\usepackage{mathrsfs}
\usepackage{mathtools}
\usepackage{extarrows}
\usepackage{graphicx,makeidx}
\usepackage{a4wide}
\usepackage{ragged2e}
\usepackage[nottoc]{tocbibind}
\usepackage{amsmath,amssymb,amsthm,amsfonts}
\usepackage{mathabx}
\usepackage[utf8]{inputenc}
\usepackage[czech]{babel}
\usepackage{relsize}
\newlength\ubwidth
\newcommand\parunderbrace[2]{\settowidth\ubwidth{$#1$}\underbrace{#1}_{\parbox{\ubwidth}{\scriptsize\RaggedRight#2}}}
\newcommand{\equal}[1]{\mathop{\overset{#1}{\resizebox{\widthof{.{\ensuremath{\mathop{\overset{#1}{\mathop{=}}}}.}}}{\heightof{=}}{$\mathop{=}$}}}} 
\usepackage[unicode,breaklinks=true,hypertexnames=false]{hyperref}
%\usepackage{color}
 
\hypersetup{
  pdftitle={01MIP - Míra a pravděpodobnost},
  pdfauthor={WIKI Skripta},
  pdfsubject={Zápisky z přednášek MIP, FJFI ČVUT},
  pdfkeywords={míra a pravděpodobnost},
  bookmarksnumbered=true,
  colorlinks=true,
  pdfpagemode={UseOutlines}
}
 
%definice českých uvozovek
\def\bq{\mbox{\kern.1ex\protect\raisebox{-1.3ex}[0pt][0pt]{''}\kern-.1ex}}
\def\eq{\mbox{\kern-.1ex``\kern.1ex}}
\gdef\uv#1{\bq #1\eq}
 
 
\renewcommand{\d}{\mathrm{d}} 
\newcommand{\m}{\mathlarger{\mathlarger{m}}} 
\newcommand{\lphi}{\raisebox{.6ex}{$\mathlarger{\mathlarger{\varphi}}$}} 
\newcommand{\E}{\mathrm{E}} %
\renewcommand{\L}{\mathrm{L}} %
\newcommand{\D}{\mathrm{D}} % 
\newcommand{\FF}{\mathrm{F}} % 
\newcommand{\PP}{\mathrm{P}} %
\newcommand{\prostor}{(\Omega,\mathcal{A},\PP)} 
\newcommand{\prostorm}{(\Omega,\mathcal{A},\mu)} 
\newcommand{\dif}{\mathrm{d}} % diferenciál
\newcommand{\me}{\mathrm{e}} % eulerovo číslo
\newcommand{\mi}{\mathrm{i}} % imaginární jednotka
\newcommand{\R}{\mathbb{R}} % množina reálných čísel
\newcommand{\Rp}{\mathbb{R}^{+}} % množina kladných reálných čísel
\newcommand{\Rm}{\mathbb{R}^{-}} % množina záporných reálných čísel
\newcommand{\Rop}{\mathbb{R}_{0}^{+}} % množina nezáporných reálných čísel
\renewcommand{\C}{\mathbb{C}} % množina komplexních čísel
\newcommand{\Z}{\mathbb{Z}} % množina celých čísel
\newcommand{\N}{\mathbb{N}} % množina přirozených čísel
\newcommand{\X}{\mathbb{X}}
\newcommand{\PEX}{\PP^\X}
\newcommand{\fex}{f_\X} 
\newcommand{\mex}{\m_\X} 
\newcommand{\FEX}{\FF_\X} 
\newcommand{\Y}{\mathbb{Y}}
\newcommand{\Aa}{\mathcal{A}}
\newcommand{\Bb}{\mathcal{B}}
\newcommand{\Nn}{\mathcal{N}}
\newcommand{\F}{\mathscr{F}}
\newcommand{\bmu}{\boldsymbol{\mu}}
\newcommand{\PEY}{\PP^\mathbb{Y}}
\newcommand{\fey}{f_\mathbb{Y}} 
\newcommand{\longequal}{=\mathrel{\mkern-4mu}=} 
\newcommand{\FEY}{\FF_\mathbb{Y}}
\newcommand{\Q}{\mathbb{Q}} % množina přirozených čísel
\newcommand{\Cc}{\mathcal{C}} % funkce třídy C (spojité)
\newcommand{\Ll}{\mathcal{L}} % funkce třídy L (integrabilní)
\newcommand{\I}{\mathcal{I}} % interval I
%\newcommand{\J}{\mathcal{J}} % interval J
\newcommand{\Ms}{\mathscr{M}} % množina M (fce stejně spojité a stejně omezené)
\newcommand{\Fs}{\mathscr{F}} % množina F (fce spojité, stejně spojité a stejně lipschitzovské
 
\newcommand{\tg}{\mathop{\mathrm{tg}}}
\newcommand{\arctg}{\mathop{\mathrm{arctg}}}
\newcommand{\sgn}{\mathop{\mathrm{sgn}}}
\renewcommand{\Re}{\mathop{\mathrm{Re}}}
\newcommand{\Ran}{\mathop{\mathrm{Ran}}}
\newcommand{\diag}{\mathop{\mathrm{diag}}}
\newcommand{\supp}{\mathop{\mathrm{supp}}}
\renewcommand{\Im}{\mathop{\mathrm{Im}}}
\newcommand{\Cov}{\mathop{\mathrm{Cov}}}
\newcommand{\Tr}{\mathop{\mathrm{Tr}}}
\newcommand{\st}{\mathop{\mathrm{st}}}
\newcommand{\LL}{\mathscr{L}}
\newcommand{\sj}{\stackrel{s.j.}{\longrightarrow}}
\newcommand{\Pto}{\stackrel{\PP}{\to}}
\newcommand{\wto}{\stackrel{w}{\to}}
\newcommand{\Dto}{\stackrel{\mathscr{D}}{\to}}
\newcommand{\Lto}{\stackrel{(\mathscr{L})}{\to}}
\newcommand{\CB}{C_B^{(0)}}
\newcommand{\CBL}{C_B^{(L)}}
\newcommand{\sjP}{\stackrel{s.j.~\PP}{\longrightarrow}}
\newcommand{\Lp}{\stackrel{L_p}{\longrightarrow}}
\newcommand{\overbar}[1]{\mkern 1mu\overline{\mkern-1mu#1\mkern-3mu}\mkern 3mu}
\newcommand{\oxn}{\overbar{\rule{0ex}{1.8ex}X_n}}
\newcommand{\oyn}{\overbar{\rule{0ex}{1.8ex}Y_n}}
\newcommand{\omn}{\overbar{\rule{0ex}{1.3ex}\mu_n}}
\newcommand{\osn}{\overbar{\rule{0ex}{1.95ex}\sigma_n^2}}
\newcommand{\posl}{(X_j)_{j=1}^{+\infty}}
\newcommand{\posln}{(X_n)_{n=1}^{+\infty}}
\newcommand{\suminfty}{\sum\limits_{j=1}^{+\infty}}
\newcommand{\suminftyk}{\sum\limits_{k=1}^{+\infty}}
\newcommand{\suminftym}{\sum\limits_{m=1}^{+\infty}}
\newcommand{\suminftykn}{\sum\limits_{k=1}^{n}}
\newcommand{\sumjn}{\sum\limits_{j=1}^{n}}


\newcommand{\salg}{$\sigma$-algebra}

\newcommand{\dom}[1]{\mathop{\mathrm{Dom} (#1)}} % definiční obor
 
\newcommand{\mat}[1]{\mathbf #1}
\newcommand{\abs}[1]{\left|#1\right|}
\newcommand{\nor}[1]{\left\|#1\right\|}
\newcommand{\Br}[1]{\Bigl(#1\Bigr)}
\newcommand{\br}[1]{\bigl(#1\bigr)}

\newcommand{\COL}[2]{\begin{pmatrix}#1\\#2\end{pmatrix}}   %sloupcový vektor 2x1 VELKÝ
\newcommand{\col}[2]{{#1 \choose #2}} %sloupcový vektor 2x1 MALÝ nebo kombinační číslo
 
\renewcommand{\epsilon}{\varepsilon}
\renewcommand{\phi}{\varphi}
\renewcommand{\rho}{\varrho}
\newcommand{\ol}{\overline}
\newcommand{\ub}{\underbrace}
\newcommand{\sm}{\smallsetminus}
\renewcommand{\emptyset}{\font\cmsy = cmsy10 at 12pt
	\hbox{\cmsy \char 59}}

 
\theoremstyle{definition}
\newtheorem{define}{Definice}[chapter]
 
\theoremstyle{plain}
\newtheorem{theorem}[define]{Věta}
\newtheorem{lemma}[define]{Lemma}
\newtheorem{dusl}[define]{Důsledek}
\newtheorem{corollary}[define]{Tvrzení}
\renewcommand{\proofname}{Důkaz}
 
\theoremstyle{remark}
\newtheorem{example}[define]{\textsc{Příklad}}
\newtheorem{remark}[define]{\textsc{Poznámka}}
 

\renewcommand{\indexname}{Rejstřík}

\frenchspacing